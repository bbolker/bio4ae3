\documentclass[12pt]{article}
\usepackage{times}
\usepackage{sober}
%\usepackage{fullpage}
\usepackage[dvips]{graphicx}
\pagestyle{headings}

\begin{document}
\newcommand{\answerboxwidth}{15cm}
%\newcommand{\answerbox}[1]{\fbox{\parbox[t][#1]{\answerboxwidth}{\
%}}}
\newcommand{\answerbox}[1]{\parbox[t][#1]{\answerboxwidth}{\ }}
\newcommand{\alphlist}{\renewcommand{\labelenumi}{\alph{enumi}.}}
\newcommand{\numlist}{\renewcommand{\labelenumi}{\arabic{enumi}.}}
\markboth%
{EEID midterm, 25 Feb 2009: \textbf{Name }}%
{EEID midterm, 25 Feb 2009: \textbf{Name }}

\begin{center}
\LARGE{ZOO 4926\S6690/6927\S6767 midterm exam, 25 Feb 2009}
\end{center}
\textbf{Please write your name on every page.}

\section{Definitions (10 points)}
Define five out of six in a phrase or sentence. \\
\begin{tabular}{lll}
a. phoresis b. symbiont c. monoxenic \\
d. trophic cascade e. herd immunity f. incidence
\end{tabular}
\newcommand{\defbox}{\answerbox{40pt}}
\numlist
\begin{enumerate}
\item{\defbox}
\item{\defbox}
\item{\defbox}
\item{\defbox}
\item{\defbox}
\end{enumerate}

\section{Short answers (20 points)}
Answer 4 out of 5.  Feel free to draw diagrams where appropriate.

\alphlist
\begin{enumerate}
\item Describe three ways that infectious disease can drive a host   population extinct (and, briefly, how each works).
\item Using the criteria ``lives in closes association with the host''  and ``routinely kills its host'', divide natural enemies into parasites, predators, grazers, and parasitoids. 
\item Give three reasons that epidemic outbreaks of a particular infectious disease would continue to occur in a population, despite the simple theoretical expectation that the outbreaks would damp out to an equilibrium over time.
\item What is the difference between exploitation, interference, and apparent competition, and how might they apply to the regulation of \emph{within-host} parasite densities?
\item Describe an example of a trait-mediated interaction that
involves parasites.
\end{enumerate}

\numlist
\newcommand{\shortanswerbox}{\answerbox{180pt}}
\begin{enumerate}
\item{\shortanswerbox}
\item{\shortanswerbox}
\item{\shortanswerbox}
\item{\shortanswerbox}
\end{enumerate}

\section{Essays (70 points)}
Answer \emph{all three}: allocate about 15 minutes to each.
Take a few minutes before you start writing
to think about the organization of your answer and what points you're
going to cover.  Use examples and/or diagrams 
to illustrate your answers where appropriate.

\alphlist
\begin{enumerate}
\item What is the difference between frequency-
and density-dependent transmission?
Why is it important?  What kinds of diseases are each/either?
What are the implications for disease persistence
and control?
\item  Blue and red tortoises have both
been present on Paynes Prairie for
a long time, but red tortoises
appear to be strongly superior competitors.
You suspect that apparent competition via
a mosquito-borne virus is responsible for
the coexistence of the tortoises.  Explain
what this all means and describe how you
would attempt to prove your hypothesis
by means of (1) a model and (2) an experiment.
\item Define "superspreaders" and explain their
importance and effects on epidemics.  Explain
how the concept of superspreaders in epidemics
is related to the concept of heterogeneity in
cells in their response to intracellular bacterial
infection.
\end{enumerate}
\newpage
\ \ 
\newpage
\ \ 
\newpage
\ \ 
\newpage
\ \ 
\end{document}
